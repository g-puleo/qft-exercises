\section{Exercise 2: Conserved charges of two scalar
fields}
\subsection{Generalization of the symmetry studied in class for one complex scalar field}
We are given the action:
\begin{equation}
    S = \int d^4 x \partial_\mu \phi_a^\star \partial^{\mu}\phi_a + \partial_\mu \phi_b^\star \partial^\mu \phi_b  - m^2 (\phi_a^\star \phi_a + \phi_b^\star \phi_b ) 
\end{equation}
Let
\begin{equation}
    \Phi = 
    \begin{pmatrix}
        \phi_a \\
        \phi_b \\
    \end{pmatrix}\in \mathbb C^2.
\end{equation}

$S$ is invariant under the following global transofmration:
\begin{equation}
    \Phi \mapsto \Phi' = e^{i\theta}\Phi , 
\end{equation}
which is an \textit{internal} symmetry.
According to Noether's theorem the corresponding conserved current is 
\begin{equation}
    j^\mu_a = -\frac{\partial \mathcal L}{\partial ( \partial_\mu \phi_k)} F_{k,a} \label{eq:conservedcurrent}
\end{equation}
where $F_k$ are found by expanding the variation of the fields to the first order in $\theta$:
\begin{align*}
\delta \phi_a = \phi'_a - \phi_a &= e^{i\theta } \phi_a - \phi_a \approx i \theta \phi_a = \theta F_0 \\
\delta \phi_a^\star = \phi_a'~^\star - \phi_a^\star &= e^{-i\theta } \phi_a^\star - \phi_a^\star \approx -i \theta \phi_a^\star = \theta F_1 \\
\delta \phi_b = \phi'_b - \phi_b &= e^{i\theta } \phi_b - \phi_b \approx i \theta \phi_b = \theta F_2 \\
\delta \phi_b^\star = \phi_b'~^\star - \phi_b^\star &= e^{-i\theta } \phi_b^\star - \phi_b^\star \approx -i \theta \phi_b^\star = \theta F_3
\end{align*}
Let $\phi_a = \phi_0$, $\phi_a ^\star= \phi_1$,
$\phi_b = \phi_2$, $\phi_b ^\star= \phi_3$,
The derivatives of the Lagrangian are 
\begin{align}
    \frac{\partial \mathcal L}{\partial(\partial_\mu \phi_a)}= \partial^\mu \phi_a^\star~~;~~~~
    \frac{\partial \mathcal L}{\partial(\partial_\mu \phi_a^\star)}= \partial^\mu \phi_a \label{eq:dLddmuphiA} \\
        \frac{\partial \mathcal L}{\partial(\partial_\mu \phi_b)}= \partial^\mu \phi_b^\star~~;~~~~
    \frac{\partial \mathcal L}{\partial(\partial_\mu \phi_b^\star)}= \partial^\mu \phi_b\label{eq:dLddmuphiB}
\end{align}
These expressions can be plugged into \eqref{eq:conservedcurrent} to find:
\begin{equation}
    j^\mu = - i \left[(\partial^\mu \phi_a^\star)\phi_a- (\partial^\mu \phi_a) \phi_a^\star + (\partial ^\mu \phi_b^\star) \phi_b - (\partial ^\mu \phi_b ) \phi_b^\star\right],
\end{equation}
the corresponding conserved charge is
\begin{equation}
    Q = \int d^3 x j^0
\end{equation}
We make sense of this by promoting the fields to operators
and normal ordering:
\begin{equation}
j^\mu = : -i \left[ (\partial^\mu \hat{\phi}_a^\dagger)\hat{\phi}_a - (\partial^\mu \hat{\phi}_a)\hat{\phi}_a^\dagger + (\partial^\mu \hat{\phi}_b^\dagger)\hat{\phi}_b - (\partial^\mu \hat{\phi}_b)\hat{\phi}_b^\dagger  \right]:
\label{eq:conserved_charge_0}
\end{equation}
Let's first compute the \(\partial^0\) derivatives for the non-dagger terms of \(\hat{\phi}_a\) and \(\hat{\phi}_b\). We will use the following expressions:

\begin{equation}
\hat{\phi}_a (x) = \frac{1}{(2\pi)^3} \int \frac{d^3 p}{\sqrt{2E_p}} \left( \hat{\alpha}_a (p) e^{-ip\cdot x} + \hat{\beta}_a^\dagger (p) e^{ip\cdot x} \right)_{\text{on shell}}
\label{eq:phia}
\end{equation}

\begin{equation}
\hat{\phi}_b (x) = \frac{1}{(2\pi)^3} \int \frac{d^3 p}{\sqrt{2E_p}} \left( \hat{\alpha}_b (p) e^{-ip\cdot x} + \hat{\beta}_b^\dagger (p) e^{ip\cdot x} \right)_{\text{on shell}}
\label{eq:phib}
\end{equation}
In the following, the fact that the integrals are evaluated on-shell is implied.

\begin{align}
\partial^0 \hat{\phi}_a (x) &=  \frac{1}{(2\pi)^3} \int \frac{d^3 p}{\sqrt{2E_p}} \left( -iE_p \hat{\alpha}_a (p) e^{-ip\cdot x} + iE_p \hat{\beta}_a^\dagger (p) e^{ip\cdot x} \right) \\
&= \frac{1}{(2\pi)^3} \int \frac{d^3 p}{\sqrt{2E_p}} \left( iE_p \left(\hat{\beta}_a^\dagger (p) e^{ip\cdot x} - \hat{\alpha}_a (p) e^{-ip\cdot x}\right) \right)
\label{eq:timederivative_of_phia}
\end{align}
Similarly, 
\begin{equation}
\partial^0 \hat{\phi}_b (x) = \frac{1}{(2\pi)^3} \int \frac{d^3 p}{\sqrt{2E_p}} \left( iE_p \left(\hat{\beta}_b^\dagger (p) e^{ip\cdot x} - \hat{\alpha}_b (p) e^{-ip\cdot x}\right) \right)
\label{eq:timederivative_of_phib}
\end{equation}
Now we can use these to evaluate the $a$ term of $j^0$, namely
\begin{equation}:(\partial^0 \hat{\phi}_a^\dagger)\hat{\phi}_a-(\partial^0 \hat{\phi}_a)\hat{\phi}_a^\dagger:\label{eq:a_term}\end{equation}
\begin{align*}
(\partial^0 \hat{\phi}_a^\dagger)(x) \hat{\phi}_a (x) &= \frac{1}{(2\pi)^6} \int \frac{d^3 p'}{\sqrt{2E_{p'}}} \int \frac{d^3 p}{\sqrt{2E_p}} \left( iE_{p'} \hat{\alpha}_a^\dagger (p') e^{ip'\cdot x} - iE_{p'} \hat{\beta}_a (p') e^{-ip'\cdot x} \right)  \\ &\times \left( \hat{\alpha}_a (p) e^{-ip\cdot x} + \hat{\beta}_a^\dagger (p) e^{ip\cdot x} \right)
\end{align*}
Expanding the product and combining the exponentials, we get
\begin{align*}
(\partial^0 \hat{\phi}_a^\dagger)(x) \hat{\phi}_a (x) &= \frac{1}{(2\pi)^6} \int \frac{d^3 p'}{\sqrt{2E_{p'}}} \int \frac{d^3 p}{\sqrt{2E_p}} \left[ iE_{p'} \hat{\alpha}_a^\dagger (p') \hat{\alpha}_a (p) e^{i(p' - p) \cdot x} +  iE_{p'} \hat{\alpha}_a^\dagger (p') \hat{\beta}_a^\dagger (p) e^{i(p' + p) \cdot x} \right .\\
& - \left .iE_{p'} \hat{\beta}_a (p') \hat{\alpha}_a (p) e^{-i(p' + p) \cdot x} - iE_{p'} \hat{\beta}_a (p') \hat{\beta}_a^\dagger (p) e^{-i(p' - p) \cdot x} \right]
\end{align*}
To compute eq. \eqref{eq:a_term} we see that we must take the last expression and subtract its own hermitian conjugate and normal-order the result:
\begin{align*}
&:(\partial^0 \hat{\phi}_a^\dagger)\hat{\phi}_a-(\partial^0 \hat{\phi}_a)\hat{\phi}_a^\dagger:  \\ 
&= \frac{1}{(2\pi)^6} \int \frac{d^3 p'}{\sqrt{2E_{p'}}} \int \frac{d^3 p}{\sqrt{2E_p}} \left[ iE_{p'} \hat{\alpha}_a^\dagger (p') \hat{\alpha}_a (p) e^{i(p' - p) \cdot x} +  iE_{p'} \hat{\alpha}_a^\dagger (p') \hat{\beta}_a^\dagger (p) e^{i(p' + p) \cdot x} \right .\\
& - iE_{p'} \hat{\beta}_a (p') \hat{\alpha}_a (p) e^{-i(p' + p) \cdot x} - iE_{p'} \hat{\beta}_a (p') \hat{\beta}_a^\dagger (p) e^{-i(p' - p) \cdot x} 
+\\
%%% hermitian conjugate
&+iE_{p'} \hat{\alpha}_a^\dagger (p') \hat{\alpha}_a (p) e^{-i(p' - p) \cdot x} +  iE_{p'} \hat{\alpha}_a (p') \hat{\beta}_a (p) e^{-i(p' + p) \cdot x}  \\
&  - \left .iE_{p'} \hat{\beta}_a^\dagger (p') \hat{\alpha}_a^\dagger (p) e^{+i(p' + p) \cdot x} - iE_{p'} \hat{\beta}_a^\dagger (p') \hat{\beta}_a (p) e^{+i(p' - p) \cdot x} 
\right]
\end{align*}
Then note that upon integration in $d^3 x$ we get Dirac deltas as follows:
\begin{equation}
\begin{split}
    \int d^3 x ~ e^{-i(p\pm p')\cdot x} = (2\pi)^3 \delta_{(3)}(\vec p\pm {\vec p}~') e^{-i (E_p\pm E_{p'})t}\\
    \int d^3 x ~ e^{+i(p\pm p')\cdot x} = (2\pi)^3 \delta_{(3)}(\vec p\pm {\vec p}~') e^{+i (E_p\pm E_{p'})t}
\end{split}
\label{eq:diracDeltas}
\end{equation}
We can plug this in and get
\begin{align*}
&\int d^3 x:(\partial^0 \hat{\phi}_a^\dagger)\hat{\phi}_a-(\partial^0 \hat{\phi}_a)\hat{\phi}_a^\dagger:  \\ 
&= :\frac{1}{(2\pi)^6} \int \frac{d^3 p'}{\sqrt{2E_{p'}}} \int \frac{d^3 p}{\sqrt{2E_p}} (2\pi)^3 \Bigg[ iE_{p'} \hat{\alpha}_a^\dagger (p') \hat{\alpha}_a (p)   \delta_{(3)}(\vec{p} - \vec{p'}) e^{i(E_{p'} - E_p)t} \\
&+  iE_{p'} \hat{\alpha}_a^\dagger (p') \hat{\beta}_a^\dagger (p)   \delta_{(3)}(\vec{p} + \vec{p'}) e^{i(E_{p'} + E_p)t} - iE_{p'} \hat{\beta}_a (p') \hat{\alpha}_a (p)   \delta_{(3)}(\vec{p} + \vec{p'}) e^{-i(E_{p'} + E_p)t} \\
&- iE_{p'} \hat{\beta}_a (p') \hat{\beta}_a^\dagger (p)   \delta_{(3)}(\vec{p} - \vec{p'}) e^{-i(E_{p'} - E_p)t} + iE_{p'} \hat{\alpha}_a^\dagger (p') \hat{\alpha}_a (p)   \delta_{(3)}(\vec{p} - \vec{p'}) e^{-i(E_{p'} - E_p)t} \\
&+ iE_{p'} \hat{\alpha}_a (p') \hat{\beta}_a (p)   \delta_{(3)}(\vec{p} + \vec{p'}) e^{-i(E_{p'} + E_p)t} - iE_{p'} \hat{\beta}_a^\dagger (p') \hat{\alpha}_a^\dagger (p)   \delta_{(3)}(\vec{p} + \vec{p'}) e^{i(E_{p'} + E_p)t} \\
&- iE_{p'} \hat{\beta}_a^\dagger (p') \hat{\beta}_a (p)   \delta_{(3)}(\vec{p} - \vec{p'}) e^{i(E_{p'} - E_p)t} \Bigg]:
\end{align*}
Integration over $\vec {p}~'$ yields 
\begin{align*}
&= \frac{1}{(2\pi)^3} \int \frac{d^3 p}{2E_p} \Bigg[ iE_p \hat{\alpha}_a^\dagger (p) \hat{\alpha}_a (p) + \cancel{ iE_p \hat{\alpha}_a^\dagger (-p) \hat{\beta}_a^\dagger (p) e^{i(2E_p)t}} \bcancel{- iE_p \hat{\beta}_a (-p) \hat{\alpha}_a (p) e^{-i(2E_p)t}} - iE_p \hat{\beta}_a (p) \hat{\beta}_a^\dagger (p) \\
&+ iE_p \hat{\alpha}_a^\dagger (p) \hat{\alpha}_a (p) +  \bcancel{ iE_p \hat{\alpha}_a (-p) \hat{\beta}_a (p) e^{-i(2E_p)t}} -     \cancel{ iE_p \hat{\beta}_a^\dagger (-p) \hat{\alpha}_a^\dagger (p) e^{i(2E_p)t}}- iE_p \hat{\beta}_a^\dagger (p) \hat{\beta}_a (p) \Bigg]
\end{align*}
Finally we have:
\begin{equation}
\begin{split}
     \int d^3 x:(\partial^0 \hat{\phi}_a^\dagger)\hat{\phi}_a-(\partial^0 \hat{\phi}_a)\hat{\phi}_a^\dagger: & = \frac{i}{(2\pi)^3} \int d^3 p \Bigg[ \hat{\alpha}_a^\dagger (p) \hat{\alpha}_a (p) - \hat{\beta}_a (p) \hat{\beta}_a^\dagger (p)    \Bigg]\\ &=i (N_{\alpha,a}-N_{\beta, a}) 
     \end{split}
\end{equation}
where we labeled the number operators for particles and antiparticles of kind $a$ as $N_{\alpha, a}$ and $N_{\beta, a}$ respectively.
The computation of the contribution from $\hat \phi_b$ to the conserved charge would be identical. 
Eventually, we resurrect the $-i$ factor in front of eq. \eqref{eq:conserved_charge_0} and get
\begin{equation}
    \hat Q = \hat N_{\alpha, a}+ \hat  N_{\alpha, b}
-  \hat N  _{\beta, a}- \hat N_{\beta,b} \end{equation}
\subsection{Symmetry under SU(2) transformation}
I observe that acting a unitary transformation $U\in {\rm SU}(2)$ on $\Phi$ will leave the inner product term $(\phi_a^\star \phi_a + \phi_b^\star \phi_b )$ unchanged. 
The mass term of the action is therefore invariant under such transformation. 
Let's check that this is also true for the partial-derivative term, which we denote as $\mathcal L^{(a)}$:
\[
    \mathcal L^{(a)} = \partial_\mu \phi_a^\star \partial^\mu \phi_a + \partial_\mu \phi_b^\star \partial^\mu \phi_b
\]
$U$ can be written as:
\[
U = \begin{pmatrix}
a & -b^\star \\ b & a^\star
 \end{pmatrix}
 \]
with $\abs{a}^2 + \abs{b}^2=1$. 
Let 
\[\begin{pmatrix}\phi_a' \\ \phi_b'\end{pmatrix} = U\Phi\]
Since $U$ is unitary, he inverse of \( U \) is
\( U^{-1} = U^\dagger \):
   \[
   U^\dagger = \begin{pmatrix}
   a^\star & b^\star \\ 
   -b & a
   \end{pmatrix}
   \]
Using the inverse transformation, we can express \(\phi_a\) and \(\phi_b\) in terms of the primed fields: 
   \[
   \Phi = U^\dagger \Phi' = \begin{pmatrix}
   a^\star & b^\star \\ 
   -b & a
   \end{pmatrix} \begin{pmatrix}
   \phi_a' \\ 
   \phi_b'
   \end{pmatrix}
   \]
   This yields:
   \[
   \phi_a = a^\star \phi_a' + b^\star \phi_b'
   \]
   \[
   \phi_b = -b \phi_a' + a \phi_b'
   \]
Substitute the expressions for \(\phi_a\) and \(\phi_b\) in terms of the primed fields:
   \[
   \partial_\mu \phi_a = \partial_\mu (a^\star \phi_a' + b^\star \phi_b') = a^\star \partial_\mu \phi_a' + b^\star \partial_\mu \phi_b'
   \]
   \[
   \partial_\mu \phi_b = \partial_\mu (-b \phi_a' + a \phi_b') = -b \partial_\mu \phi_a' + a \partial_\mu \phi_b'
   \]
   and
   \[
   \partial_\mu \phi_a^\star = (a^\star \partial_\mu \phi_a' + b^\star \partial_\mu \phi_b')^\star = a \partial_\mu \phi_a'^\star + b \partial_\mu \phi_b'^\star
   \]
   \[
   \partial_\mu \phi_b^\star = (-b \partial_\mu \phi_a' + a \partial_\mu \phi_b')^\star = -b^\star \partial_\mu \phi_a'^\star + a^\star \partial_\mu \phi_b'^\star
   \]
   Now we substitute these into \(\mathcal L^{(a)}\):
   \[
   \mathcal L^{(a)} = (a \partial_\mu \phi_a'^\star + b \partial_\mu \phi_b'^\star)(a^\star \partial^\mu \phi_a' + b^\star \partial^\mu \phi_b') + (-b^\star \partial_\mu \phi_a'^\star + a^\star \partial_\mu \phi_b'^\star)(-b \partial^\mu \phi_a' + a \partial^\mu \phi_b')
   \]
   and simplify each term,
   \[
   \mathcal L^{(a)} = |a|^2 \partial_\mu \phi_a'^\star \partial^\mu \phi_a' + a b^\star \partial_\mu \phi_a'^\star \partial^\mu \phi_b' + a^\star b \partial_\mu \phi_b'^\star \partial^\mu \phi_a' + |b|^2 \partial_\mu \phi_b'^\star \partial^\mu \phi_b'
   \]
   \[
   + |b|^2 \partial_\mu \phi_a'^\star \partial^\mu \phi_a' - a b^\star \partial_\mu \phi_a'^\star \partial^\mu \phi_b' - a^\star b \partial_\mu \phi_b'^\star \partial^\mu \phi_a' + |a|^2 \partial_\mu \phi_b'^\star \partial^\mu \phi_b'
   \]
   Notice that the mixed terms cancel out, therefore:
   \[
   \mathcal L^{(a)} = (|a|^2 + |b|^2) \partial_\mu \phi_a'^\star \partial^\mu \phi_a' + (|b|^2 + |a|^2) \partial_\mu \phi_b'^\star \partial^\mu \phi_b' 
   \]
   Since \(|a|^2 + |b|^2 = 1\):
   \[
   \mathcal L^{(a)} = \partial_\mu \phi_a'^\star \partial^\mu \phi_a' + \partial_\mu \phi_b'^\star \partial^\mu \phi_b'
   \]
Thus, the Lagrangian \( \mathcal L^{(a)} \) retains its original form when expressed in terms of the primed fields.
So, $U$ is also an internal symmetry, and we now want to express the conserved currents according to eq. \eqref{eq:conservedcurrent}. Now we have a 3-parameter symmetry so there will be 3 conserved currents $j^\mu_k$ with $k\in\{1,2,3\}$. 
We start by writing the ${\rm SU}(2)$ transformation as 
\begin{equation}
    U=\exp (-i \theta_k \sigma^{(k)}/2) \approx  \mathbb I - \frac{i\theta_k }{2}\sigma^{(k)}, \label{eq:lieSU2}
\end{equation}
where $\sigma^{(k)}$ are the Pauli matrices and we have expanded to lowest order in $\theta_k$. 
We want to express the variation of the fields as
\begin{equation}
\delta \phi_\ell = F_{\ell,k} \theta^k\label{eq:defF}
\end{equation}
now $k\in\{1,2,3\}$ and $\ell$ identifies the field as in the previous section. 
Let's evaluate \eqref{eq:lieSU2} explicitly. Recall the Pauli matrices \(\sigma^{(k)}\) are:
   \[
   \sigma^{(1)} = \sigma^x = \begin{pmatrix} 0 & 1 \\ 1 & 0 \end{pmatrix}, \quad \sigma^{(2)} = \sigma^y = \begin{pmatrix} 0 & -i \\ i & 0 \end{pmatrix}, \quad \sigma^{(3)} = \sigma^z = \begin{pmatrix} 1 & 0 \\ 0 & -1 \end{pmatrix}
   \]

Now, we compute the expression \(i \theta_k \sigma^{(k)} \Phi\) for each \(k\) and then do the sum. 
\begin{align}
i \theta_1 \sigma^{(1)} \Phi &= i \theta_1 \begin{pmatrix} 0 & 1 \\ 1 & 0 \end{pmatrix} \begin{pmatrix} \phi_a \\ \phi_b \end{pmatrix} = i \theta_1 \begin{pmatrix} \phi_b \\ \phi_a \end{pmatrix}
\\
i \theta_2 \sigma^{(2)} \Phi &= i \theta_2 \begin{pmatrix} 0 & -i \\ i & 0 \end{pmatrix} \begin{pmatrix} \phi_a \\ \phi_b \end{pmatrix} = i \theta_2 \begin{pmatrix} -i \phi_b \\ i \phi_a \end{pmatrix} = \theta_2 \begin{pmatrix} \phi_b \\ -\phi_a \end{pmatrix}
\\
i \theta_3 \sigma^{(3)} \Phi &= i \theta_3 \begin{pmatrix} 1 & 0 \\ 0 & -1 \end{pmatrix} \begin{pmatrix} \phi_a \\ \phi_b \end{pmatrix} = i \theta_3 \begin{pmatrix} \phi_a \\ -\phi_b \end{pmatrix}
\end{align}
Summing these up, we get:
\[
i \theta_k \sigma^{(k)} \Phi = i \theta_1 \begin{pmatrix} \phi_b \\ \phi_a \end{pmatrix} + \theta_2 \begin{pmatrix} \phi_b \\ -\phi_a \end{pmatrix} + i \theta_3 \begin{pmatrix} \phi_a \\ -\phi_b \end{pmatrix}
\]
Combining the terms and using \eqref{eq:lieSU2}, we get
\[
\begin{pmatrix}
    \delta \phi_a \\
    \delta \phi_b 
\end{pmatrix}
=
-\frac{1}{2}i \theta_k \sigma^{(k)} \Phi = -\frac{1}{2}\begin{pmatrix} i \theta_1 \phi_b + \theta_2 \phi_b + i \theta_3 \phi_a \\ i \theta_1 \phi_a - \theta_2 \phi_a - i \theta_3 \phi_b \end{pmatrix}
\]
Similarly the variation of the conjugate fields is gotten by complex-conjugating eq. \eqref{eq:lieSU2}, and noticing that  $(\sigma^{(2)})^\star=-\sigma^{(2)}$, while the other two Pauli matrices are real. We get
\begin{equation*}
    \begin{pmatrix}
    \delta \phi_a^\star \\
    \delta \phi_b ^\star
\end{pmatrix}
=
+\frac{1}{2}i \theta_k( \sigma^{(k)} )^\star \Phi^\star = \frac{1}{2}\begin{pmatrix} i \theta_1 \phi_b^\star - \theta_2 \phi_b^\star + i \theta_3 \phi_a^\star \\ i \theta_1 \phi_a^\star + \theta_2 \phi_a^\star - i \theta_3 \phi_b^\star \end{pmatrix}
\end{equation*}
Comparing the last two expressions with equation \eqref{eq:defF}
\begin{align}
F_{0,1} &= -\frac{i}{2} \phi_b, & F_{0,2} &= -\frac{1}{2} \phi_b, & F_{0,3} &= -\frac{i}{2} \phi_a, \\
F_{2,1} &= -\frac{i}{2} \phi_a, & F_{2,2} &= \frac{1}{2} \phi_a, & F_{2,3} &= \frac{i}{2} \phi_b, \\
F_{1,1} &= \frac{i}{2} \phi_b^\star, & F_{1,2} &= -\frac{1}{2} \phi_b^\star, & F_{1,3} &= \frac{i}{2} \phi_a^\star, \\
F_{3,1} &= \frac{i}{2} \phi_a^\star, & F_{3,2} &= \frac{1}{2} \phi_a^\star, & F_{3,3} &= -\frac{i}{2} \phi_b^\star.
\end{align}

% \begin{equation}
%    j^\mu_1 = \frac{i}{2}\left[ - \phi_b (\partial^\mu \phi_a^\star)  + \phi_b (\partial^\mu \phi_a)  - \phi_a (\partial^\mu \phi_b^\star) + \phi_a (\partial^\mu \phi_b) \right]\\
   
% \end{equation}
Given the conserved current \( j^\mu_\ell \) and the Lagrangian derivatives, we need to find the expressions for the three conserved currents \( j^\mu_1 \), \( j^\mu_2 \), and \( j^\mu_3 \). The conserved current is given by:

\[ j^\mu_\ell = -\frac{\partial \mathcal L}{\partial ( \partial_\mu \phi_k)} F_{k,\ell} \]
For \(\ell = 1\):
   \[
   \begin{aligned}
   j^\mu_1 &= -\left( \frac{\partial \mathcal{L}}{\partial (\partial_\mu \phi_0)} F_{0,1} + \frac{\partial \mathcal{L}}{\partial (\partial_\mu \phi_1)} F_{1,1} + \frac{\partial \mathcal{L}}{\partial (\partial_\mu \phi_2)} F_{2,1} + \frac{\partial \mathcal{L}}{\partial (\partial_\mu \phi_3)} F_{3,1} \right) \\
   &= -\left( \partial^\mu \phi_a^\star \cdot \left( -\frac{i}{2} \phi_b \right) + \partial^\mu \phi_a \cdot \frac{i}{2} \phi_b^\star+ \partial^\mu \phi_b^\star \cdot \left( -\frac{i}{2} \phi_a \right) + \partial^\mu \phi_b \cdot \frac{i}{2} \phi_a^\star \right) \\
   &= -\left( -\frac{i}{2} \partial^\mu \phi_a^\star \phi_b + \frac{i}{2} \partial^\mu \phi_a \phi_b^\star - \frac{i}{2} \partial^\mu \phi_b^\star \phi_a + \frac{i}{2} \partial^\mu \phi_b \phi_a^\star \right) \\
   &= \frac{i}{2} \left( \partial^\mu \phi_a^\star \phi_b - \partial^\mu \phi_a \phi_b^\star + \partial^\mu \phi_b^\star \phi_a - \partial^\mu \phi_b \phi_a^\star \right).
   \end{aligned}
   \]
For \(\ell = 2\):
   \[
   \begin{aligned}
   j^\mu_2 &= -\left( \frac{\partial \mathcal{L}}{\partial (\partial_\mu \phi_0)} F_{0,2} + \frac{\partial \mathcal{L}}{\partial (\partial_\mu \phi_1)} F_{1,2} + \frac{\partial \mathcal{L}}{\partial (\partial_\mu \phi_2)} F_{2,2} + \frac{\partial \mathcal{L}}{\partial (\partial_\mu \phi_3)} F_{3,2} \right) \\
   &= -\left( \partial^\mu \phi_a^\star \cdot \left( -\frac{1}{2} \phi_b\right) + \partial^\mu \phi_a \cdot \left( -\frac{1}{2} \phi_b^\star \right) + \partial^\mu \phi_b^\star \cdot \frac{1}{2} \phi_a + \partial^\mu \phi_b \cdot \frac{1}{2} \phi_a^\star \right) \\
   &= -\left( -\frac{1}{2} \partial^\mu \phi_a^\star \phi_b - \frac{1}{2} \partial^\mu \phi_a \phi_b^\star + \frac{1}{2} \partial^\mu \phi_b^\star \phi_a + \frac{1}{2} \partial^\mu \phi_b \phi_a^\star \right) \\
   &= \frac{1}{2} \left( \partial^\mu \phi_a^\star \phi_b + \partial^\mu \phi_a \phi_b^\star - \partial^\mu \phi_b^\star \phi_a - \partial^\mu \phi_b \phi_a^\star \right).
   \end{aligned}
   \]
For \(\ell = 3\):
   \[
   \begin{aligned}
   j^\mu_3 &= -\left( \frac{\partial \mathcal{L}}{\partial (\partial_\mu \phi_0)} F_{0,3} + \frac{\partial \mathcal{L}}{\partial (\partial_\mu \phi_1)} F_{1,3} + \frac{\partial \mathcal{L}}{\partial (\partial_\mu \phi_2)} F_{2,3} + \frac{\partial \mathcal{L}}{\partial (\partial_\mu \phi_3)} F_{3,3} \right) \\
   &= -\left( \partial^\mu \phi_a^\star \cdot \left( -\frac{i}{2} \phi_a \right) + \partial^\mu \phi_a \cdot \frac{i}{2} \phi_a^\star + \partial^\mu \phi_b^\star \cdot \frac{i}{2} \phi_b + \partial^\mu \phi_b \cdot \left( -\frac{i}{2} \phi_b^\star \right) \right) \\
   &= -\left( -\frac{i}{2} \partial^\mu \phi_a^\star \phi_a + \frac{i}{2} \partial^\mu \phi_a \phi_a^\star + \frac{i}{2} \partial^\mu \phi_b^\star \phi_b - \frac{i}{2} \partial^\mu \phi_b \phi_b^\star \right) \\
   &= \frac{i}{2} \left( \partial^\mu \phi_a^\star \phi_a - \partial^\mu \phi_a \phi_a^\star - \partial^\mu \phi_b^\star \phi_b + \partial^\mu \phi_b \phi_b^\star \right).
   \end{aligned}
   \]
The conserved charges are gotten by setting $j=0$. It's useful to introduce the conjugate momenta $\pi_a$ and $\pi_b$
\begin{align*} \pi_a = \frac{\partial \mathcal L}{\partial(\partial_0 \phi_a)}= \partial^0 \phi_a^\star~~;~~~~
    \pi_a^\star = \frac{\partial \mathcal L}{\partial(\partial_0 \phi_a^\star)}= \partial^0 \phi_a  \\
       \pi_b =  \frac{\partial \mathcal L}{\partial(\partial_0 \phi_b)}= \partial^0 \phi_b^\star~~;~~~~
   \pi_b^\star =  \frac{\partial \mathcal L}{\partial(\partial_0 \phi_b^\star)}= \partial^0 \phi_b
\end{align*}
so the time-component of the currents can be written more compactly:
\begin{align*}
j^0_1 &= \frac{i}{2} \left( \pi_a \phi_b - \pi_a^\star \phi_b^\star + \pi_b \phi_a - \pi_b^\star \phi_a^\star \right)\\ 
j^0_2 &= \frac{1}{2} \left( \pi_a \phi_b + \pi_a^\star \phi_b^\star - \pi_b \phi_a - \pi_b^\star \phi_a^\star \right)\\
j^0_3 &= \frac{i}{2} \left( \pi_a \phi_a - \pi_a^\star \phi_a^\star - \pi_b \phi_b + \pi_b^\star \phi_b^\star \right)
\end{align*}
and it's possible to notice that this can be further simplified to 
\[
j_k^0 = -\frac{i}{2} ( \phi^\star _\mu (\sigma^k)_{\mu\nu} \pi_\nu^\star - \pi_\mu(\sigma^k)_{\mu\nu}\phi_\nu )
\]
where now $\mu$ and $\nu$ are indices that run over the two fields $a$ and $b$. When $a$ and $b$ are index of pauli matrix, they refer to 1 and 2, respectively.
 This expression is neater to work with and it is equal (up to a sign) to the one in Peskin, Schr\"{o}der's book, problem 2.2 (d).
In order to find the commutation relation between the charges, we promote the fields to operators as above, 
and use equations \eqref{eq:phia}, \eqref{eq:phib}, \eqref{eq:timederivative_of_phia}, \eqref{eq:timederivative_of_phib} to decompose fields using 
creation and annihilation operators.
I will stick to Peskin and Schr\"{o}der's result without the minus sign. 
\[
\hat j_k^0 = \frac{i}{2} :( \hat \phi^\dagger _\mu (\sigma^k)_{\mu\nu}( \partial^0 \hat \phi_\nu) - (\partial^0 \hat \phi_\mu^\dagger)(\sigma^k)_{\mu\nu}\hat \phi_\nu ):
\]
We focus on the first term, and forget the $i/2$ factor and normal ordering for a moment. We will restore them at the end.
We have
\begin{equation}
\begin{split}
\nonumber
    \phi^\dagger _\mu (\sigma^k)_{\mu\nu}( \partial^0 \phi_\nu) &= \frac{1}{(2\pi)^6} \int \frac{d^3 p}{\sqrt{2E_p}} \int \frac{d^3 p'}{\sqrt{2E_{p'}}} \left( \hat{\alpha}_\mu^\dagger (p) e^{ip\cdot x} + \hat{\beta}_\mu (p) e^{-ip\cdot x} \right)\\
    &\times \sigma^k_{\mu\nu} iE_{p'} \left(\hat{\beta}_\nu^\dagger (p') e^{ip'\cdot x} - \hat{\alpha}_\nu (p') e^{-ip'\cdot x}\right)
    \\
    &=\frac{i}{(2\pi)^6} \int \frac{d^3 p}{\sqrt{2E_p}} \int \frac{d^3 p'}{\sqrt{2E_{p'}}} \bigg[ E_{p'} \hat{\alpha}_\mu^\dagger (p) \sigma^k_{\mu\nu} \hat{\beta}_\nu^\dagger (p') e^{i(p+p')\cdot x} \\
    &\quad - E_{p'} \hat{\alpha}_\mu^\dagger (p) \sigma^k_{\mu\nu} \hat{\alpha}_\nu (p') e^{i(p-p')\cdot x} \\
    &\quad + E_{p'} \hat{\beta}_\mu (p) \sigma^k_{\mu\nu} \hat{\beta}_\nu^\dagger (p') e^{-i(p-p')\cdot x} \\
    &\quad - E_{p'} \hat{\beta}_\mu (p) \sigma^k_{\mu\nu} \hat{\alpha}_\nu (p') e^{-i(p+p')\cdot x} \bigg]
\end{split}
\end{equation}
Similarly, the second term:
\begin{equation}
    \nonumber
\begin{split}
    (\partial^0 \hat{\phi}_\mu^\dagger)(\sigma^k)_{\mu\nu}\hat{\phi}_\nu &= \frac{i}{(2\pi)^6} \int \frac{d^3 p}{\sqrt{2E_p}} \int \frac{d^3 p'}{\sqrt{2E_{p'}}} \bigg[ - E_p \hat{\beta}_\mu (p) \sigma^k_{\mu\nu} \hat{\alpha}_\nu (p') e^{-i(p+p')\cdot x} \\
    &\quad - E_p \hat{\beta}_\mu (p) \sigma^k_{\mu\nu} \hat{\beta}_\nu^\dagger (p') e^{-i(p-p')\cdot x} \\
    &\quad + E_p \hat{\alpha}_\mu^\dagger (p) \sigma^k_{\mu\nu} \hat{\alpha}_\nu (p') e^{i(p-p')\cdot x} \\
    &\quad + E_p \hat{\alpha}_\mu^\dagger (p) \sigma^k_{\mu\nu} \hat{\beta}_\nu^\dagger (p') e^{i(p+p')\cdot x} \bigg]
\end{split}
\end{equation}
We proceed by subtracting the two terms  and integrating over $d^3 x$,  we get Dirac deltas as described by equation \eqref{eq:diracDeltas}, then we carry out the integration over $d^3 p'$ and use that $E_p=E_{-p}$ to simplify the resulting integrals. 
\begin{align*}
    \hat Q^k &= \frac{i}{(2\pi)^6}  \int d^3 x \left\{  \int \frac{d^3 p}{\sqrt{2E_p}} \int \frac{d^3 p'}{\sqrt{2E_{p'}}} E_{p'}  \sigma^k_{\mu\nu} \bigg[  \hat{\alpha}_\mu^\dagger (p) \hat{\beta}_\nu^\dagger (p') e^{i(p+p')\cdot x} \right. \\
    &\quad -  \hat{\alpha}_\mu^\dagger (p) \hat{\alpha}_\nu (p') e^{i(p-p')\cdot x} + \quad +  \hat{\beta}_\mu (p) \hat{\beta}_\nu^\dagger (p') e^{-i(p-p')\cdot x} \\
    &\quad -  \hat{\beta}_\mu (p) \hat{\alpha}_\nu (p') e^{-i(p+p')\cdot x} + \hat{\beta}_\mu (p) \hat{\alpha}_\nu (p') e^{-i(p+p')\cdot x} \\
    &\quad + \hat{\beta}_\mu (p) \hat{\beta}_\nu^\dagger (p') e^{-i(p-p')\cdot x}  - \hat{\alpha}_\mu^\dagger (p) \hat{\alpha}_\nu (p') e^{i(p-p')\cdot x} \\
    &\quad \left. - \hat{\alpha}_\mu^\dagger (p) \hat{\beta}_\nu^\dagger (p') e^{i(p+p')\cdot x} \bigg]\right\}\\
    &= \frac{i}{(2\pi)^3} \int \frac{d^3 p}{\sqrt{2E_p}} \int \frac{d^3 p'}{\sqrt{2E_{p'}}} E_{p'} \sigma^k_{\mu\nu} \bigg\{ \delta_{(3)}(\vec{p}+\vec{p}') e^{i(E_p+E_{p'})t} \hat{\alpha}_\mu^\dagger (p) \hat{\beta}_\nu^\dagger (p') \\
    &\quad - \delta_{(3)}(\vec{p}-\vec{p}') e^{i(E_p-E_{p'})t} \hat{\alpha}_\mu^\dagger (p) \hat{\alpha}_\nu (p')+ \delta_{(3)}(\vec{p}-\vec{p}') e^{-i(E_p-E_{p'})t} \hat{\beta}_\mu (p) \hat{\beta}_\nu^\dagger (p') \\
    &\quad - \delta_{(3)}(\vec{p}+\vec{p}') e^{-i(E_p+E_{p'})t} \hat{\beta}_\mu (p) \hat{\alpha}_\nu (p') + \delta_{(3)}(\vec{p}+\vec{p}') e^{-i(E_p+E_{p'})t} \hat{\beta}_\mu (p) \hat{\alpha}_\nu (p') \\
    &\quad + \delta_{(3)}(\vec{p}-\vec{p}') e^{-i(E_p-E_{p'})t} \hat{\beta}_\mu (p) \hat{\beta}_\nu^\dagger (p') - \delta_{(3)}(\vec{p}-\vec{p}') e^{i(E_p-E_{p'})t} \hat{\alpha}_\mu^\dagger (p) \hat{\alpha}_\nu (p') \\
    &\quad - \delta_{(3)}(\vec{p}+\vec{p}') e^{i(E_p+E_{p'})t} \hat{\alpha}_\mu^\dagger (p) \hat{\beta}_\nu^\dagger (p') \bigg\}\\
    %%  Now integration over p' is done
    &= \frac{i}{(2\pi)^3} \int \frac{d^3 p}{2E_p} E_p \sigma^k_{\mu\nu} \bigg\{ e^{i(E_p+E_p)t} \hat{\alpha}_\mu^\dagger (p) \hat{\beta}_\nu^\dagger (-p) \\
    &\quad - e^{i(E_p-E_p)t} \hat{\alpha}_\mu^\dagger (p) \hat{\alpha}_\nu (p) + e^{-i(E_p-E_p)t} \hat{\beta}_\mu (p) \hat{\beta}_\nu^\dagger (p) \\
    &\quad - e^{-i(E_p+E_p)t} \hat{\beta}_\mu (p) \hat{\alpha}_\nu (-p)  + e^{-i(E_p+E_p)t} \hat{\beta}_\mu (p) \hat{\alpha}_\nu (-p) \\
    &\quad + e^{-i(E_p-E_p)t} \hat{\beta}_\mu (p) \hat{\beta}_\nu^\dagger (p) - e^{i(E_p-E_p)t} \hat{\alpha}_\mu^\dagger (p) \hat{\alpha}_\nu (p) \\
    &\quad - e^{i(E_p+E_p)t} \hat{\alpha}_\mu^\dagger (p) \hat{\beta}_\nu^\dagger (-p) \bigg\}.
\end{align*}
We can simplify this expression further, by canceling out some terms, and combining like terms:
\[
\begin{split}
    \hat{Q}^k &= \frac{i}{(2\pi)^3} \int \frac{d^3 p}{2E_p} E_p \sigma^k_{\mu\nu} \bigg\{ \cancel{e^{2iE_pt} \hat{\alpha}_\mu^\dagger (p) \hat{\beta}_\nu^\dagger (-p)}  - \hat{\alpha}_\mu^\dagger (p) \hat{\alpha}_\nu (p) + \hat{\beta}_\mu (p) \hat{\beta}_\nu^\dagger (p) \\
    &\cancel{\quad - e^{-2iE_pt} \hat{\beta}_\mu (p) \hat{\alpha}_\nu (-p)} +\cancel{ e^{-2iE_pt} \hat{\beta}_\mu (p) \hat{\alpha}_\nu (-p)}  + \hat{\beta}_\mu (p) \hat{\beta}_\nu^\dagger (p)  - \hat{\alpha}_\mu^\dagger (p) \hat{\alpha}_\nu (p) \\
    &\quad - \cancel{e^{2iE_pt} \hat{\alpha}_\mu^\dagger (p) \hat{\beta}_\nu^\dagger (-p)} \bigg\}\\
    &=\frac{i}{(2\pi)^3} \int d^3 p\sigma^k_{\mu\nu} \bigg( \hat{\beta}_\mu (p) \hat{\beta}_\nu^\dagger (p) - \hat{\alpha}_\mu^\dagger (p) \hat{\alpha}_\nu (p) \bigg).
\end{split}
\]
Restoring the $i/2$ factor and normal ordering, we get the final expression for the conserved charge:
\[
    \hat Q_k =-\frac{1}{2} \frac{1}{(2\pi)^3} \int d^3 p\sigma^k_{\mu\nu} \bigg( \hat{\beta}_\nu^\dagger (p) \hat{\beta}_\mu (p)  - \hat{\alpha}_\mu^\dagger (p) \hat{\alpha}_\nu (p) \bigg).
\]
Now we want to compute the commutation relations between these charges.
I will take out the factors in front and restore them in the end, for simplicity of writing. 
\begin{align*}
[\hat{Q}_k, \hat{Q}_\ell] &= \int d^3 p \int d^3 q \, \sigma^k_{\mu\nu} \sigma^\ell_{\rho\eta} \left[ \left( \hat{\beta}_\nu^\dagger (p) \hat{\beta}_\mu (p) - \hat{\alpha}_\mu^\dagger (p) \hat{\alpha}_\nu (p) \right), \left( \hat{\beta}_\eta^\dagger (q) \hat{\beta}_\rho (q) - \hat{\alpha}_\rho^\dagger (q) \hat{\alpha}_\eta (q) \right) \right]\\
   &=\int d^3 p \int d^3 q \, \sigma^k_{\mu\nu} \sigma^\ell_{\rho\eta} \bigg( \left[ \hat{\beta}_\nu^\dagger (p) \hat{\beta}_\mu (p), \hat{\beta}_\eta^\dagger (q) \hat{\beta}_\rho (q) \right] - \left[ \hat{\beta}_\nu^\dagger (p) \hat{\beta}_\mu (p), \hat{\alpha}_\rho^\dagger (q) \hat{\alpha}_\eta (q) \right] \\
    &\quad - \left[ \hat{\alpha}_\mu^\dagger (p) \hat{\alpha}_\nu (p), \hat{\beta}_\eta^\dagger (q) \hat{\beta}_\rho (q) \right] + \left[ \hat{\alpha}_\mu^\dagger (p) \hat{\alpha}_\nu (p), \hat{\alpha}_\rho^\dagger (q) \hat{\alpha}_\eta (q) \right] \bigg)
    \label{eq:lotofcommutators}
\end{align*}
There are now a lot of commutators that we can simplify using the following commutation relations:
\begin{align}
[\hat{\alpha}_\mu (p), \hat{\alpha}_\nu^\dagger (q)] &= [\hat{\beta}_\mu (p), \hat{\beta}_\nu^\dagger (q)] = (2\pi)^3 \delta_{(3)}(\vec{p}-\vec{q}) \delta_{\mu\nu} \\
[\hat{\alpha}_\mu (p), \hat{\beta}_\nu (q)] &= [\hat{\alpha}_\mu (p), \hat{\beta}_\nu^\dagger (q)] = [\hat{\alpha}_\mu^\dagger (p), \hat{\beta}_\nu (q)] = [\hat{\alpha}_\mu^\dagger (p), \hat{\beta}_\nu^\dagger (q)] = 0
\end{align}
The second and third commutator in equation \eqref{eq:lotofcommutators} vanish, and the first and fourth commutators are non-zero. We can now simplify them using this property:
% Noting that \( - e^{-2iE_pt} \hat{\beta}_\mu (p) \hat{\alpha}_\nu (-p) + e^{-2iE_pt} \hat{\beta}_\mu (p) \hat{\alpha}_\nu (-p) = 0 \):
\begin{equation*}
    [AB, CD] =A[B, C]D + [A, C]BD + CA[B, D] + C[A, D]B 
\end{equation*}
We get
\[ 
\begin{split}
    %% First commutator
    &\left[ \hat{\beta}_\nu^\dagger (p) \hat{\beta}_\mu (p), \hat{\beta}_\eta^\dagger (q) \hat{\beta}_\rho (q) \right] = \hat{\beta}_\nu^\dagger (p) \left[ \hat{\beta}_\mu (p), \hat{\beta}_\eta^\dagger (q) \right] \hat{\beta}_\rho (q) + \left[ \hat{\beta}_\nu^\dagger (p), \hat{\beta}_\eta^\dagger (q) \right] \hat{\beta}_\mu (p) \hat{\beta}_\rho (q) \\
    &\quad + \hat{\beta}_\eta^\dagger (q)  \hat{\beta}_\nu^\dagger (p)\left[ \hat{\beta}_\mu (p),\hat{\beta}_\rho (q) \right]  +  \hat{\beta}_\eta^\dagger (q) \left[ \hat{\beta}_\nu^\dagger (p), \hat{\beta}_\rho (q) \right] \hat{\beta}_\mu (p) \\
    &= (2\pi)^3 \delta_{(3)}(\vec{p}-\vec{q}) \delta_{\mu\eta} \hat{\beta}_\nu^\dagger (p) \hat{\beta}_\rho (q) - (2\pi)^3 \delta_{(3)}(\vec{p}-\vec{q}) \delta_{\nu\rho}  \hat{\beta}_\eta^\dagger (q) \hat{\beta}_\mu (p) \\
    %% Now the other commutator
    &\left[ \hat{\alpha}_\mu^\dagger (p) \hat{\alpha}_\nu (p), \hat{\alpha}_\rho^\dagger (q) \hat{\alpha}_\eta (q) \right] = \hat{\alpha}_\mu^\dagger (p) \left[ \hat{\alpha}_\nu (p), \hat{\alpha}_\rho^\dagger (q) \right] \hat{\alpha}_\eta (q) + \left[ \hat{\alpha}_\mu^\dagger (p), \hat{\alpha}_\rho^\dagger (q) \right] \hat{\alpha}_\nu (p) \hat{\alpha}_\eta (q) \\
    &\quad + \hat{\alpha}_\rho^\dagger (q)  \hat{\alpha}_\mu^\dagger (p)\left[\hat{\alpha}_\nu (p), \hat{\alpha}_\eta (q) \right]  +  \hat{\alpha}_\rho^\dagger (q) \left[ \hat{\alpha}_\mu^\dagger (p), \hat{\alpha}_\eta (q) \right] \hat{\alpha}_\nu (p) \\
    &= (2\pi)^3 \delta_{(3)}(\vec{p}-\vec{q}) \delta_{\nu\rho} \hat{\alpha}_\mu^\dagger (p) \hat{\alpha}_\eta (q) - (2\pi)^3 \delta_{(3)}(\vec{p}-\vec{q}) \delta_{\mu\eta}  \hat{\alpha}_\rho^\dagger (q)\hat{\alpha}_\nu (p) 
\end{split}
\]
So, the commutator can be simplified as follows:
\[ 
\begin{split}
    [\hat{Q}_k, \hat{Q}_\ell] &= \int d^3 p \int d^3 q \, \sigma^k_{\mu\nu} \sigma^\ell_{\rho\eta} \bigg( \left[ \hat{\beta}_\nu^\dagger (p) \hat{\beta}_\mu (p), \hat{\beta}_\eta^\dagger (q) \hat{\beta}_\rho (q) \right] + \left[ \hat{\alpha}_\mu^\dagger (p) \hat{\alpha}_\nu (p), \hat{\alpha}_\rho^\dagger (q) \hat{\alpha}_\eta (q) \right] \bigg) \\
    %%
    &= \int d^3 p \int d^3 q \, \sigma^k_{\mu\nu} \sigma^\ell_{\rho\eta} \bigg( (2\pi)^3 \delta_{(3)}(\vec{p}-\vec{q}) \delta_{\mu\eta} \hat{\beta}_\nu^\dagger (p) \hat{\beta}_\rho (q) - (2\pi)^3 \delta_{(3)}(\vec{p}-\vec{q}) \delta_{\nu\rho} \hat{\beta}_\eta^\dagger (q) \hat{\beta}_\mu (p)   \\
    &\quad - (2\pi)^3 \delta_{(3)}(\vec{p}-\vec{q}) \delta_{\nu\rho} \hat{\alpha}_\mu^\dagger (p) \hat{\alpha}_\eta (q) + (2\pi)^3 \delta_{(3)}(\vec{p}-\vec{q}) \delta_{\mu\eta}\hat{\alpha}_\rho^\dagger (q) \hat{\alpha}_\nu (p)  \bigg)\\
    %%
    &= (2\pi)^3\int d^3 p \,\bigg( \sigma^k_{\mu\nu} \sigma^\ell_{\rho\eta} \delta_{\mu\eta} \hat{\beta}_\nu^\dagger (p) \hat{\beta}_\rho (p) 
    -\sigma^k_{\mu\nu} \sigma^\ell_{\rho\eta} \delta_{\nu\rho}\hat{\beta}_\eta^\dagger (p)\hat{\beta}_\mu (p)   \\
    &\quad + \sigma^k_{\mu\nu} \sigma^\ell_{\rho\eta} \delta_{\nu\rho} \hat{\alpha}_\mu^\dagger (p) \hat{\alpha}_\eta (p) 
    - \sigma^k_{\mu\nu} \sigma^\ell_{\rho\eta} \delta_{\mu\eta}\hat{\alpha}_\rho^\dagger (p)  \hat{\alpha}_\nu (p)  \bigg)\\
    %
    %
    %
    &= (2\pi)^3 \int d^3 p \,\bigg( \sigma^k_{\mu\nu} \sigma^\ell_{\rho\mu} \hat{\beta}_\nu^\dagger (p) \hat{\beta}_\rho (p) 
    -\sigma^k_{\mu\nu} \sigma^\ell_{\nu\eta}  \hat{\beta}_\eta^\dagger (p)\hat{\beta}_\mu (p)  \\
    &\quad + \sigma^k_{\mu\nu} \sigma^\ell_{\nu\eta}  \hat{\alpha}_\mu^\dagger (p) \hat{\alpha}_\eta (p) 
    - \sigma^k_{\mu\nu} \sigma^\ell_{\rho\mu}  \hat{\alpha}_\rho^\dagger (p) \hat{\alpha}_\nu (p) \bigg) \\
    % FINAL EXTRAVAGANZA %%%%%%%%%%%%%%%%%%%%%%%%%%%%%%%%%%%%%%%%%%%%%%%%%%%%%%%%%%%%%%%%%%%%%%%%%%%%%%%%%%%%%
    &= (2\pi)^3 \int d^3 p \,\bigg(  {(\sigma^\ell \sigma^k)}_{\rho\nu} \hat{\beta}_\nu^\dagger (p) \hat{\beta}_\rho (p) 
    -{(\sigma^k \sigma^\ell)}_{\mu\eta}  \hat{\beta}_\eta^\dagger (p)\hat{\beta}_\mu (p)  \\
    &\quad + (\sigma^k \sigma^\ell)_{\mu\eta}  \hat{\alpha}_\mu^\dagger (p) \hat{\alpha}_\eta (p) 
    - (\sigma^\ell \sigma^k)_{\rho\nu}  \hat{\alpha}_\rho^\dagger (p) \hat{\alpha}_\nu (p) \bigg) \\
    %
    %
    &= (2\pi)^3 \int d^3 p \,\bigg(  {[\sigma^\ell,\sigma^k]}_{\rho\nu} \hat{\beta}_\nu^\dagger (p) \hat{\beta}_\rho (p)  
    - [\sigma^\ell \sigma^k]_{\rho\nu}  \hat{\alpha}_\rho^\dagger (p) \hat{\alpha}_\nu (p) \bigg) \\
    %
    &=(2\pi)^3 {[\sigma^\ell,\sigma^k]}_{\rho\nu} \int d^3 p \,   \bigg( \hat{\beta}_\nu^\dagger (p) \hat{\beta}_\rho (p)  
    -   \hat{\alpha}_\rho^\dagger (p) \hat{\alpha}_\nu (p) \bigg) 
\end{split}
\]
Now I restore the pre-factor $-1/(2(2\pi)^3)$ and use the Pauli matrix commutation relation:
\begin{align*}
[\hat Q_k, \hat Q_\ell] &=
 \left(-\frac{1}{2(2\pi)^3}\right)^2 (2\pi^3) 2i \epsilon_{\ell k j} 
    \sigma^j_{\rho\nu} \int d^3 p \, \left( \hat{\beta}_\nu^\dagger (p) \hat{\beta}_\rho (p) - \hat{\alpha}_\rho^\dagger (p) \hat{\alpha}_\nu (p) \right) \\
    &= i \epsilon_{ k\ell j} \sigma^j_{\rho\nu} \left( -\frac{1}{2(2\pi)^3} \right) \int d^3 p \, \left( \hat{\beta}_\nu^\dagger (p) \hat{\beta}_\rho (p) - \hat{\alpha}_\rho^\dagger (p) \hat{\alpha}_\nu (p) \right)\\ 
    &= i \epsilon_{ k\ell j}  \hat Q_j.
\end{align*}
Which means that the conserved charges obey the SU(2) algebra.
% Simplify further using the Kronecker deltas
% \[ 
% \begin{split}
%     [\hat{Q}_k, \hat{Q}_\ell] &= \int d^3 p \, \bigg( \sigma^k_{\mu\nu} \sigma^\ell_{\mu\nu} \hat{\beta}_\nu^\dagger (p) \hat{\beta}_\nu (p) - \sigma^k_{\mu\nu} \sigma^\ell_{\nu\mu} \hat{\beta}_\nu^\dagger (p) \hat{\beta}_\mu (p) \\
%     &\quad - \sigma^k_{\mu\nu} \sigma^\ell_{\nu\mu} \hat{\alpha}_\mu^\dagger (p) \hat{\alpha}_\nu (p) + \sigma^k_{\mu\nu} \sigma^\ell_{\mu\nu} \hat{\alpha}_\nu^\dagger (p) \hat{\alpha}_\nu (p) \bigg).
% \end{split}
% \[
% \begin{split}
%     \hat{Q}^k &= \frac{i}{(2\pi)^3} \int \frac{d^3 p}{\sqrt{2E_p}} E_p \sigma^k_{\mu\nu} \bigg\{ e^{2iE_pt} \hat{\alpha}_\mu^\dagger (p) \hat{\beta}_\nu^\dagger (-p) - \hat{\alpha}_\mu^\dagger (p) \hat{\alpha}_\nu (p) \\
%     &\quad + \hat{\beta}_\mu (p) \hat{\beta}_\nu^\dagger (p) + \hat{\beta}_\mu (p) \hat{\beta}_\nu^\dagger (p) \\
%     &\quad - \hat{\alpha}_\mu^\dagger (p) \hat{\alpha}_\nu (p) - e^{2iE_pt} \hat{\alpha}_\mu^\dagger (p) \hat{\beta}_\nu^\dagger (-p) \bigg\}.
% \end{split}
% \]


% \[
% \begin{split}
%     \hat{Q}^k &= \frac{i}{(2\pi)^3} \int \frac{d^3 p}{\sqrt{2E_p}} E_p \sigma^k_{\mu\nu} \bigg\{ 2\hat{\beta}_\mu

%  (p) \hat{\beta}_\nu^\dagger (p) - 2\hat{\alpha}_\mu^\dagger (p) \hat{\alpha}_\nu (p) \bigg\}.
% \end{split}
% \]

% Thus, the final expression is:

% \[
% \begin{split}
%     \hat{Q}^k &= \frac{2i}{(2\pi)^3} \int \frac{d^3 p}{\sqrt{2E_p}} E_p \sigma^k_{\mu\nu} \bigg( \hat{\beta}_\mu (p) \hat{\beta}_\nu^\dagger (p) - \hat{\alpha}_\mu^\dagger (p) \hat{\alpha}_\nu (p) \bigg).
% \end{split}
% \]
